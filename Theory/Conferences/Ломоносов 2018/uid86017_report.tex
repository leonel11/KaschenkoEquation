\nonstopmode

\documentclass[a4paper, 12pt, oneside]{ncc}
\usepackage[warn]{mathtext}          % русские буквы в формулах, с предупреждением
\usepackage{ucs}                     % Unicode input support
\usepackage[T2A]{fontenc}            % внутренняя кодировка  TeX
\usepackage[utf8x]{inputenc}         % кодовая страница документа
\usepackage[english, russian]{babel} % локализация и переносы
\usepackage{indentfirst}             % русский стиль: отступ первого абзаца раздела
\usepackage{misccorr}                % точка в номерах заголовков
\usepackage{cmap}                    % русский поиск в pdf
\usepackage{graphicx}                % Работа с графикой \includegraphics{}
\usepackage{caption}                 % Работа с подписями для фигур, таблиц и пр.
\usepackage{soul}                    % Разряженный текст \so{} и подчеркивание \ul{}
\usepackage{soulutf8}                % Поддержка UTF8 в soul
\usepackage{fancyhdr}                % Для работы с колонтитулами
\usepackage{multirow}                % Аналог multicolumn для строк
\usepackage{ltxtable}                % Микс tabularx и longtable
\usepackage{paralist}                % Списки с отступом только в первой строчке
\usepackage[shortcuts]{extdash}      % Переносы в словах с дефисами
\usepackage[perpage]{footmisc}       % Нумерация сносок на каждой странице с 1
\usepackage{amsmath}
\usepackage{amsfonts}
\usepackage{amssymb}
% Задаем отступы: слева 30 мм, справа 10 мм, сверху до колонтитула 10 мм снизу 25 мм
\usepackage[a4paper, top=20mm, left=30mm, right=10mm, bottom=25mm]{geometry}
\usepackage{xcolor}
\usepackage{hyperref}

% цвета для гиперссылок
\definecolor{linkcolor}{HTML}{340DAB} % цвет ссылок
\definecolor{urlcolor}{HTML}{340DAB} % цвет гиперссылок


\hypersetup{unicode=true, linkcolor=linkcolor, urlcolor=urlcolor, colorlinks=true, breaklinks=true}

\def\EUR{\,\euro}

\pagestyle{fancy}
\fancyhead{}
\fancyhead[LE]{\textit{ Конференция <<Ломоносов 2018>>}}
\fancyhead[LO]{\textit{ Конференция <<Ломоносов 2018>>}}
\fancyfoot{}
\fancyfoot[RE,RO]{\thepage}
\renewcommand{\headrulewidth}{0pt}
\renewcommand{\footrulewidth}{0pt}

\title{ Колебательная потеря устойчивости нулевого решения в одной нелинейной краевой задаче с запаздыванием }
\author{ Ивановский Леонид Игоревич }

\makeatletter
\setlength{\@fptop}{0pt}
\setlength{\@fpbot}{0pt plus 1fil}
\makeatother

\begin{document}
\begin{flushright}
Секция <<Дифференциальные уравнения, динамические системы и оптимальное управление>>
\end{flushright}

\begin{center}
\textbf{Колебательная потеря устойчивости нулевого решения в одной нелинейной краевой задаче с запаздыванием}
\end{center}

\begin{center}
\textbf{Научный руководитель -- Глызин Сергей Дмитриевич }\\
\vspace{0.2cm}
                                                                                                        }

    \textit{Ивановский Л.И.$^{1}$, Куксёнок И.С.$^{2}$}}\\
    \small{
    1 - Ярославский государственный университет им. П.Г. Демидова, Ярославль, Россия; 2 - Национальный исследовательский ядерный университет <<МИФИ>>, Москва, Россия}}\\
\\
\end{center}

Рассмотрим нелинейную краевую задачу c запаздыванием: 
\begin{equation}\label{eq1}
	\dot{u} = u'' + \gamma u(t - h) - u^3,
\end{equation}
\begin{equation}\label{eq2}
	u'(0, t) = 0, \quad u'(0, t) = \alpha u(x_0, t - \tau),
\end{equation}
где $ \alpha, \, \gamma \in \mathbb{R}, \, \tau, \, h \geqslant 0, \, x_0 \in [0, 1] $.

Нулевое состояние равновесия краевой задачи \eqref{eq1}, \eqref{eq2} колебательно теряет устойчивость в том случае, если все собственные числа лежат в левой комплексной полуплоскости, а одна из пар находится на мнимой оси.

Рассмотрим разложение решения краевой задачи \eqref{eq1}, \eqref{eq2} в нормальной форме по степеням малого параметра $ \varepsilon $
\begin{equation}\label{eq3}
	\sum\limits_{j=1}^{\infty} \varepsilon^{\frac{j+1}{2}} u_j(t, s, x),
\end{equation}
где $ s = \varepsilon t $ -- медленное время, а $ u_0 $ имеет вид
$$ u_0 = z(s) e^{iwt} w(x) + \overline{z(s)} e^{-iwt} \overline{w(x)}. $$

Подстановка разложения \eqref{eq3} в задачу \eqref{eq1}, \eqref{eq2} приводит к системе последовательно разрешимых краевых задач с модифицированными краевыми условиями. Эта система позволяет получить уравнение на амплитуду колебаний нулевого решения вида
\begin{equation}\label{eq4}
	\dot{z} = \phi z + dz|z|^2.
\end{equation}
Для исследования экспоненциально-орбитально устойчивого цикла, описываемого формулой \eqref{eq3}, достаточно изучить зависимость действительных частей параметров $ \phi $ и $ d $ дифференциального уравнения \eqref{eq4} от значений параметров краевой задачи \eqref{eq1}, \eqref{eq2}.

Исследование выполнено за счет гранта Российского научного фонда (проект №14-21-00158).

\begin{center}\textbf{Источники и литература}\end{center}
\begin{enumerate}
\item Кащенко С.А. О бифуркациях при малых возмущениях в логистическом уравнении с запаздыванием // Моделирование и анализ информационных систем. 2017. Т. 24, №2. С. 168 – 185.{\sloppy

}
\end{enumerate}
\\


\end{document}