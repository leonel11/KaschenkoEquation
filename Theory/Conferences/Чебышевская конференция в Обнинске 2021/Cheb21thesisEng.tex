\documentclass[12pt]{report}
\usepackage[utf8x]{inputenc}
\usepackage[T2A]{fontenc}
\usepackage{ucs}
\usepackage[english,russian]{babel}

%\usepackage[math]{pscyr}
\usepackage{amssymb}
\usepackage{amsmath}
\usepackage{hyperref}

\usepackage{cheb21}

\begin{document}
\selectlanguage{English}

\renewcommand{\AuthorNoA}{Galkin V. A.}
\renewcommand{\AuthorNoB}{Morgun D. A.}
%\renewcommand{\AuthorNoC}{}

\renewcommand{\authorlist}{%
\AuthorNoA, %
\AuthorNoB%, %
%\AuthorNoC%, %
}

\renewcommand{\rustitle}{Abstract Template}
\PrerenderUnicode{\rustitle}
\chebsectionrus{\authorlist}{\rustitle}
\AddIndex{\AuthorNoA}
\AddIndex{\AuthorNoB}
%\AddIndex{\AuthorNoC}

\begin{center}
\textbf{\AuthorNoA${}^{1,a}$, \AuthorNoB${}^{1,b}$}

${}^{1}$ \textit{Surgut Office, Scientific Research Institute for System Analysis of the Russian Academy of Sciences, Surgut, Russian Federation}

${}^{a}$ \textit{val-gal@yandex.ru}, ${}^{b}$ \textit{morgun\_da@office.niisi.tech}
 
\end{center}

\chebkeyeng{requirements for presentation, template}

This study is supported by XXXX grants YY-YY-YYYYY, ZZ-ZZ-ZZZZZZ.

\noindent {}

Please visit the conference website http://chebconf.ru for the abstract layout guidelines and an abstract template. The abstract, including a list of references, shall not exceed one A4 page.

\textbf{Abstract structure}

1. Title (no abbreviations  in the title, please.) No period in the end.

2. Last name and initials of the authors. The author's name is followed by a footnote indicating their affiliation (no abbreviations, no department names), city, country, and e-mail address.

3. Keywords (3--6).

4. Acknowledgments. This may include grants that supported the work (if any.)

5. Body text.

6. References.

Equations referenced in the text are numbered in Arabic numerals (1). Figures should be saved in the figs subfolder of the abstract folder (the folder name structure is ``FirstAuthorIOyear-month-day'', where FirstAuthorIO is the last name and initials of the first author.)

\begin{equation} \label{eq:Cheb19SurMathGalkin__1_} 
\frac{\partial \textbf{H}}{\partial t} +\left(\textbf{v}\cdot \nabla \right)\textbf{H}=\left(\textbf{H}\cdot \nabla \right)\textbf{v}+\mu _{m} \Delta \textbf{H}.
\end{equation} 

References in the text are in brackets (for example, [2]). The list should be arranged in the order of citation in the text (Vancouver style.)

For the reference list entries please use the ``Cite this article'' sections in journals and online publications. 


%\noindent Литература

\begin{thebibliography}{9}

\bibitem{bib:Cheb21Galkin01eng}
Kirillova O. V., Popova N. G., Skalaban A. V. et al. \textit{Guidelines for Scientific Journal Web Site as a Presentation Tool for Domestic and International Audience.} Yekaterinburg: Urals Univ. Press; 2018. 92 p. (In Russ.) Available at: 10.24069/B978-5-7996-2332-6.

\bibitem{bib:Cheb21Galkin02eng}
Rew D. Thoughts on Journal Titles for Russian Editors and Publishers. \textit{Science Editor and Publisher.} 2016;1(1-4):46–47. (In Russ.) DOI: 10.24069/2542-0267-2016-1-4-46-47.

\bibitem{bib:Cheb21Galkin03eng}
\textit{Scopus Content Selection and Advisory Board.} Available at: \url{https://www.elsevier.com/solutions/scopus/how-scopus-works/content/scopus-content-selection-and-advisory-board}.

\end{thebibliography}

\end{document}
