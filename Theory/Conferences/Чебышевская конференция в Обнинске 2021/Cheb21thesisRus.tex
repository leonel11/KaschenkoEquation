\documentclass[12pt]{report}
\usepackage[utf8x]{inputenc}
\usepackage[T2A]{fontenc}
\usepackage{ucs}
\usepackage[english,russian]{babel}

%\usepackage[math]{pscyr}
\usepackage{amssymb}
\usepackage{amsmath}
\usepackage{hyperref}

\usepackage{cheb21}

\begin{document}
\selectlanguage{russian}

\renewcommand{\AuthorNoA}{Галкин В. А.}
\renewcommand{\AuthorNoB}{Моргун Д. А.}
%\renewcommand{\AuthorNoC}{}

\renewcommand{\authorlist}{%
\AuthorNoA, %
\AuthorNoB%, %
%\AuthorNoC%, %
}

\renewcommand{\rustitle}{Пример-шаблон тезиса}
\PrerenderUnicode{\rustitle}
\chebsectionrus{\authorlist}{\rustitle}
\AddIndex{\AuthorNoA}
\AddIndex{\AuthorNoB}
%\AddIndex{\AuthorNoC}

\begin{center}
\textbf{\AuthorNoA${}^{1,\textit{а}}$, \AuthorNoB${}^{1,\textit{б}}$}

${}^{1}$ \textit{Сургутский филиал Федерального научного центра «Научно-исследовательский институт системных исследований Российской академии наук», г. Сургут,\\ Российская Федерация}

${}^{\textit{а}}$ \textit{val-gal@yandex.ru}, ${}^{\textit{б}}$ \textit{morgun\_da@office.niisi.tech}
 
\end{center}

\chebkeyrus{требования к оформлению, шаблон}

Работа поддержана грантами XXXX № YY-YY-YYYYY, № ZZ-ZZ-ZZZZZZ.

\noindent {}

Требования к оформлению тезисов и примеры-шаблоны тезисов размещены на сайте конференции http://chebconf.ru. Объём тезисов, включая список литературы, 
--- не более одной страницы формата А4 для тезисов на русском языке и не более одной страницы формата A4 для тезисов на английском языке.


\textbf{Структура тезиса}

1. Название статьи (аббревиатура в названии недопустима). Точка после названия не
ставится.

2. Фамилия (полностью), имя, отчество (инициалы) автора. После фамилии автора следует сноска 
с указанием места работы (без аббревиатур и упоминания ведомственной принадлежности), города, страны и электронного адреса. 

3. Ключевые слова (3-6) на русском языке.

4. Благодарности. Здесь могут упоминаться гранты, при поддержке которых работа
выполнялась (при наличии).

5. Текст тезиса. 

6. Литература. 

Нумерация формул выполняется арабскими цифрами (1), (2) и только для тех формул, на которые имеются ссылки в тексте.
Рисунки должны храниться в подкаталоге figs каталога с материалами тезиса (<<FirstAuthorIOгод-месяц-день>>, где FirstAuthorIO --- фамилия и инициалы первого автора).

\begin{equation} \label{eq:Cheb19SurMathGalkin__1_} 
\frac{\partial \textbf{H}}{\partial t} +\left(\textbf{v}\cdot \nabla \right)\textbf{H}=\left(\textbf{H}\cdot \nabla \right)\textbf{v}+\mu _{m} \Delta \textbf{H}.
\end{equation} 

Библиографические ссылки в тексте статьи выделяют квадратными скобками,
указывая номер источника в списке литературы (например, [2]). Источники приводятся в порядке упоминания в тексте (Ванкуверский стиль).

При составлении библиографических описаний статей следует пользоваться разделами Для цитирования (For Citation) в журналах и на сайтах
сетевых изданий.


%\noindent Литература

\begin{thebibliography}{9}

\bibitem{bib:Cheb21Galkin01}
Кириллова О. В., Попова Н. Г., Скалабан А. В. и др. {\it Рекомендации по подготовке сайта научного журнала для преставления издания российскому и международному
сообществу.} Екатеринбург: Изд-во Урал. ун-та; 2018. 92 с. Режим доступа: 10.24069/B978-5-7996-2332-6.

\bibitem{bib:Cheb21Galkin02}
Рю Д. Размышления по поводу названия журнала: в помощь российским редакторам и издателям. {\it Научный редактор и издатель.} 2016;1(1-4):46–47. DOI: 10.24069/2542-0267-2016-
1-4-46-47.

\bibitem{bib:Cheb21Galkin03}
{\it Охват контента Scopus.} Режим доступа: \url{http://elsevierscience.ru/files/Scopus_Content_Guide_Rus_2017.pdf}.

\end{thebibliography}

\end{document}
