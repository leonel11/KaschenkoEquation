\documentclass[12pt]{report}
\usepackage[utf8x]{inputenc}
\usepackage[T2A]{fontenc}
\usepackage{ucs}
\usepackage[english,russian]{babel}

%\usepackage[math]{pscyr}
\usepackage{amssymb}
\usepackage{amsmath}
\usepackage{hyperref}

\usepackage{cheb21}

\begin{document}
\selectlanguage{russian}

\renewcommand{\AuthorNoA}{Ивановский Л. И.}

\renewcommand{\authorlist}{%
\AuthorNoA, %
}

\renewcommand{\rustitle}{Устойчивость нулевого решения одной параболической краевой задачи с дополнительной внутренней связью}
\PrerenderUnicode{\rustitle}
\chebsectionrus{\authorlist}{\rustitle}
\AddIndex{\AuthorNoA}

\begin{center}
\textbf{\AuthorNoA${}^{1,\textit{а}}$}

${}^{1}$ \textit{Ярославский государственный университет им. П.Г. Демидова, г. Ярославль,\\ Российская Федерация}

${}^{\textit{а}}$ \textit{leon19unknown@gmail.com}
 
\end{center}

\chebkeyrus{параболическая краевая задача, нулевое состояние равновесия, потеря устойчивости, бифуркации}

Работа поддержана грантом РФФИ № 18-29-10055.

\noindent {}

Рассмотрим параболическую краевую задачу
\begin{equation} \label{eq:Ivanovskii_1} 
\dot u = u'' + \gamma u - u^3,
\end{equation} 
с дополнительной внутренней связью во втором краевом условии
\begin{equation} \label{eq:Ivanovskii_2} 
u'(0, t) \, = 0, \qquad u'(1, t) \, = \alpha\,u(x_0, t),
\end{equation} 
где функция $ u(x, t)$ --- гладкая при $t\ge 0$ и $x\in[0,1]$, параметры $ \alpha, \gamma \in \mathbb{R} $, а величина $ x_0 \in [0, 1) $. Для задачи \eqref{eq:Ivanovskii_1}, \eqref{eq:Ivanovskii_2} можно выделить два способа потери устойчивости нулевого состояния равновесия --- дивергентный, когда в спектре устойчивости появляется нулевое значение, и колебательный, соответствующий случаю перехода пары комплексно сопряженных собственных значений из левой комплексной полуплоскости на мнимую ось. Задача исследования состояла в изучении характера потери устойчивости нулевого решения краевой задачи \eqref{eq:Ivanovskii_1}, \eqref{eq:Ivanovskii_2}, а именно в поиске критических значений параметров $ \alpha, \; \gamma $ и $ x_0 $ и получении асимптотических формул для режимов, ответвляющихся от нулевого состояния равновесия.

Поскольку получить нужные критические значения параметров с использованием одного лишь аналитического аппарата довольно затруднительно, исследование осуществлялось численно. В результате были найдены критические значения параметров $ \alpha, \; \gamma $ и $ x_0 $, при которых происходят различные бифуркации нулевого состояния равновесия краевой задачи \eqref{eq:Ivanovskii_1}, \eqref{eq:Ivanovskii_2}. При значениях параметра $ \alpha $, близких к критическим, была построена нормальная форма и на ее основе были определены условия появления пространственно неоднородных устойчивых состояний равновесия в одном случае и циклов в другом.

%\noindent Литература

\begin{thebibliography}{9}

\bibitem{bib:Clyzin}
Глызин С. Д., Колесов А. Ю., Розов Н. Х. Диффузионный хаос и его инвариантные числовые характеристики. {\it ТМФ. } 2020;203(1):10–25. DOI: https://doi.org/10.4213/tmf9824. 

\bibitem{bib:Ivanovskii}
Ивановский Л. И. Динамика одной системы диффузионно связанных дифференциальных уравнений с дополнительной внутренней связью. {\it Известия высших учебных заведений. Поволжский регион. Физико-математические науки.} 2020;3(55):15–30. DOI: 10.21685/2072-3040-2020-3-2.

\end{thebibliography}

\end{document}
