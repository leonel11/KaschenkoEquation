
\nonstopmode

\documentclass[a4paper, 12pt, oneside]{ncc}
\usepackage[warn]{mathtext}          % русские буквы в формулах, с предупреждением
\usepackage{ucs}                     % Unicode input support
\usepackage[T2A]{fontenc}            % внутренняя кодировка  TeX
\usepackage[utf8x]{inputenc}         % кодовая страница документа
\usepackage[english, russian]{babel} % локализация и переносы
\usepackage{indentfirst}             % русский стиль: отступ первого абзаца раздела
\usepackage{misccorr}                % точка в номерах заголовков
\usepackage{cmap}                    % русский поиск в pdf
\usepackage{graphicx}                % Работа с графикой \includegraphics{}
\usepackage{caption}                 % Работа с подписями для фигур, таблиц и пр.
\usepackage{soul}                    % Разряженный текст \so{} и подчеркивание \ul{}
\usepackage{soulutf8}                % Поддержка UTF8 в soul
\usepackage{fancyhdr}                % Для работы с колонтитулами
\usepackage{multirow}                % Аналог multicolumn для строк
\usepackage{ltxtable}                % Микс tabularx и longtable
\usepackage{paralist}                % Списки с отступом только в первой строчке
\usepackage[shortcuts]{extdash}      % Переносы в словах с дефисами
\usepackage[perpage]{footmisc}       % Нумерация сносок на каждой странице с 1
\usepackage{amsmath}
\usepackage{amsfonts}
\usepackage{amssymb}
% Задаем отступы: слева 30 мм, справа 10 мм, сверху до колонтитула 10 мм снизу 25 мм
\usepackage[a4paper, top=20mm, left=30mm, right=10mm, bottom=25mm]{geometry}
\usepackage{xcolor}
\usepackage{hyperref}

% цвета для гиперссылок
\definecolor{linkcolor}{HTML}{340DAB} % цвет ссылок
\definecolor{urlcolor}{HTML}{340DAB} % цвет гиперссылок

\hypersetup{unicode=true, linkcolor=linkcolor, urlcolor=urlcolor, colorlinks=true, breaklinks=true}

\def\EUR{\,\euro}

\pagestyle{fancy}
\fancyhead{}
\fancyhead[LE]{\textit{ Конференция <<Ломоносов 2019>>}}
\fancyhead[LO]{\textit{ Конференция <<Ломоносов 2019>>}}
\fancyfoot{}
\fancyfoot[RE,RO]{\thepage}
\renewcommand{\headrulewidth}{0pt}
\renewcommand{\footrulewidth}{0pt}

\title{ Потеря устойчивости нулевого состояния равновесия одной краевой задачи с линейным отклонением в краевом условии }
\author{ Ивановский Леонид Игоревич }

\begin{document}
\begin{flushright}
Секция <<Дифференциальные уравнения, динамические системы и оптимальное управление>>
\end{flushright}

\begin{center}
\textbf{Потеря устойчивости нулевого состояния равновесия одной краевой задачи с линейным отклонением в краевом условии}
\end{center}


\begin{center}
\textbf{\textit{Ивановский Леонид Игоревич}}\\
\textit{Аспирант}\\
Ярославский государственный университет им. П.Г. Демидова, Ярославль, Россия\\
\textit{E-mail: leon19unknown@gmail.com}
\\
\end{center}


Рассмотрим краевую задачу c линейным отклонением в одном из краевых условий
\begin{equation}\label{ivanovsky-eq1}
	\dot u = u'' + \gamma u - u^3,	
\end{equation}
\begin{equation}\label{ivanovsky-eq2}	
	u'(0, t) \, = 0, \qquad u'(1, t) \, = \alpha\,u(x_0, t),
\end{equation}
для которой параметры $ \alpha, \gamma \in \mathbb{R} $, а $ x_0 \in [0,1] $. Краевая задача \eqref{ivanovsky-eq1}, \eqref{ivanovsky-eq2} очевидным образом имеет нулевое решение. В зависимости от значений параметров, это решение может быть устойчивым или неустойчивым. Представляет интерес определить условие устойчивости нулевого состояния равновесия и выяснить какие решения от него ответвляются при ее потере. В данном случае основными способами потери устойчивости являются дивергентный, когда в спектре устойчивости состояния равновесия появляется нулевое значение, и колебательный, соответствующий случаю выхода пары собственных значений  на мнимую ось.

Для выяснения устойчивости нулевого решения для линеаризованной в нуле краевой задачи \eqref{ivanovsky-eq1}, \eqref{ivanovsky-eq2} выполняется стандартная эйлерова замена вида $ u(x, t) = e^{\lambda t} \, v(x) $ . Тогда для функции $ v(x) $ получается следующая задача на собственные значения:
$$ v'' + (\gamma - \lambda)v = 0, $$
$$ v'(0) \, = 0, \qquad v'(1) \, = \alpha\,v(x_0), $$
При решении этой краевой задачи получается характеристическое уравнение вида
\begin{equation}\label{ivanovsky-eq3}
	\sqrt{-\gamma + \lambda} \; \mbox{sh} \sqrt{-\gamma + \lambda} \, = \, \alpha. 
\end{equation}
Здесь выяснение устойчивости нулевого состояния равновесия сталкивается со следующими трудностями: можно найти значения параметров, при которых корни характеристического уравнения \eqref{ivanovsky-eq3} пересекают мнимую ось. Однако в таком случае не удается доказать, что все остальные корни будут лежать слева от мнимой оси. В связи с этим в комплексе будут применяться аналитические и численные методы для решения данной задачи.

Для уравнения \eqref{ivanovsky-eq3} выясняется важный вопрос о критических значениях $\alpha_{cr}(\gamma)$, при которых корни уравнения \eqref{ivanovsky-eq3} выходят на мнимую ось, в зависимости от того, каким образом, дивергентным или колебательным, нулевое состояние равновесие краевой задачи \eqref{ivanovsky-eq1}, \eqref{ivanovsky-eq2} теряет свою устойчивость. Для изучения фазового портрета краевой задачи \eqref{ivanovsky-eq1}, \eqref{ivanovsky-eq2} используется нормальная форма, которая получается в результате разложения решения краевой задачи \eqref{ivanovsky-eq1}, \eqref{ivanovsky-eq2} по степеням малого параметра, косвенно характеризующего собой отклонение нулевого состояния равновесия от положения равновесия.

\begin{center}\textbf{Источники и литература}\end{center}
\begin{enumerate}
\item Кащенко С.\,А. ``О бифуркациях при малых возмущениях в логистическом уравнении с запаздыванием'', \textit{Моделирование и анализ информационных систем}, \textbf{24}:2 (2017), с. 168 – 185.{\sloppy

}
\item Хэссард Б., Казаринов Н., Вэн И. ``Теория и приложения бифуркации рождения цикла'', \textit{М.: Мир}, 1985. - 280 с.{\sloppy

}
\end{enumerate}
\\


\end{document}
