\documentclass[12pt]{article}
\usepackage[T2A]{fontenc}
\usepackage[cp1251]{inputenc}
\renewcommand{\baselinestretch}{1.5}
\textwidth=158mm
\textheight=232mm
\voffset=-24mm
\pagestyle{empty}
\begin{document}
\centerline{\large\bf Bifurcations of zero balance state in one boundary-value problem }
\centerline{\large\bf with linear deviation in boundary condition }
\medskip
\centerline{\bf Glyzin S.D., Ivanovsky L.I. }
\medskip
\centerline{\it P.G. Demidov Yaroslavl State University}
\centerline{\it glyzin.s@gmail.com, leon19unknown@gmail.com}
\bigskip
Let us consider dynamic properties of boundary-value problem
$$ \dot u = u'' + \gamma u - u^3, \eqno(1)  $$
with linear deviation in one boundary condition
$$ u'(0, t) \, = 0, \qquad u'(1, t) \, = \alpha\,u(x_0, t), \eqno(2) $$
where $ u(x, t)$ is a smooth function, $t\ge 0, \; x\in[0,1], $ parameters $ \alpha, \gamma $ are real numbers and $ x_0 \in [0, 1) $. 

In boundary-value problem (1), (2) there are implemented two cases of stability loss of zero balance state --- divergent, when the zero value appears in the spectrum of stability, and oscillating, when the spectrum of stability has a pair of complex eigenvalues with zero real parts. Our task of research was to find critical values of parameters $ \alpha, \gamma $ and detect regimes which derive from zero balance state of boundary-value problem (1), (2).

As a result of numerical research there were found areas of values $ \gamma $ and $ \alpha $, where various bifurcations of zero solution took place. When parameter $ \alpha $ was close to the critical value, by means of normal form there were determined conditions of appearance of nonuniform balance states and cycles.

This work was supported by the Russian Science Foundation (project \textnumero 18-29-10055).

\end{document}