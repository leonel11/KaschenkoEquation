\documentclass[fullscreen=true, unicode, bookmarks=false]{beamer}
\usepackage[T2A]{fontenc}
\usepackage[utf8]{inputenc}
\usepackage[english, russian]{babel}
\usepackage{amsmath}
\usepackage{amsmath,amsfonts,amssymb}
\usepackage[export]{adjustbox}
\usepackage{textgreek}
\newtheorem{rustheorem}{Утверждение }
\sloppy

\setbeamertemplate{navigation symbols}{}

\usetheme{Madrid}

\usecolortheme{whale}

\usefonttheme{professionalfonts} % default family is serif

\setbeamertemplate{footline}{\hspace*{.5cm}\scriptsize{\insertshorttitle
\hspace*{50pt} \hfill\hspace*{.5cm}}\vspace{5pt}} 

\setbeamercolor{bibliography entry author}{fg=black}

\title[]{ {\huge Бифуркационные особенности одной нелинейной краевой задачи с интегральным отклонением в краевом условии } } 
\date{ }  

\begin{document}

\begin{frame}
\titlepage
\end{frame} 

\begin{frame}
\frametitle{ Краевая задача с интегральным отклонением в краевом условии }
 
\begin{equation}
	\dot u = u'' + \gamma u - u^3,	
\end{equation}

\begin{equation}
	u'(0, t) \, = 0, \qquad u'(1, t) \, = \alpha \int\limits_{0}^{1} u(t, y) dy,
\end{equation}

\bigskip

$$ \alpha, \gamma \in \mathbb{R}, \qquad t \geqslant 0, \qquad x \in [0,1]. $$

\end{frame}

\begin{frame}
\frametitle{ Линеаризованная краевая задача }
 
\begin{equation}
	\dot u = u'' + \gamma u,	
\end{equation}

\begin{equation}	
	u'(0, t) \, = 0, \qquad u'(1, t) \, = \alpha_{cr}\int\limits_{0}^{1} u(t, y) dy.
\end{equation}

\end{frame}

\begin{frame}
\frametitle{ Задача на собственные значения }
 
$$ u(x, t) = e^{\lambda t} \, v(x). $$

\bigskip
 
\begin{equation}
	v'' + (\gamma - \lambda)v = 0,	
\end{equation}

\begin{equation}	
	v'(0) \, = 0, \qquad v'(1) \, = \alpha_{cr}\int\limits_{0}^{1} v(y) dy.
\end{equation}

\bigskip

$$ \mu = \sqrt{-\gamma + \lambda}, $$

$$ v(x) = c \ch  \mu x, \quad c \in \mathbb{R}. $$

\end{frame}

\begin{frame}
\frametitle{ Потеря устойчивости нулевого состояния равновесия }
 
\begin{equation}
	\alpha_{cr} = \mu^2 = -\gamma + \lambda,
\end{equation}

\bigskip

\begin{itemize}

\item { $ \lambda = 0: \; \mu = \sqrt{-\gamma}, $ 
}

\begin{equation}
	\alpha_u = -\gamma.
\end{equation}

\end{itemize}	

\end{frame}

\begin{frame}
\frametitle{ Локальный анализ краевой задачи }

\begin{equation}
	u = \sqrt{\varepsilon}u_0 + \varepsilon u_1 + \varepsilon^{\frac{3}{2}} u_2 + O(\varepsilon^2),
\end{equation}

\bigskip

$$ \varepsilon = \alpha - \alpha_u, $$

$$ \varepsilon \ll 1, \quad s = \varepsilon t. $$

\end{frame}

\begin{frame}
\frametitle{ Случай дивергентной потери устойчивости }

\begin{equation}
	\dot u_0 = u_0'' + \gamma u_0,
\end{equation}
\begin{equation}
	u_0'(0, t) \, = 0, \qquad u_0'(1, t) = \alpha_u\int\limits_{0}^{1} u_0(s, y) dy,
\end{equation}

$$ u_0 = \rho(s) \ch \sqrt{-\gamma} x. $$

\bigskip

\begin{equation}
	\dot u_2 + \frac{\partial u_0}{\partial s} = u_2'' + \gamma u_2 - u_0^3,
\end{equation}
\begin{equation}
	u_2'(0, t) \, = 0, \qquad u_2'(1, t) = \alpha_u\int\limits_{0}^{1} u_2(t, y) dy \, + \, \int\limits_{0}^{1} u_0(s, y) dy,
\end{equation}

\end{frame}

\begin{frame}
\frametitle{ Случай дивергентной потери устойчивости }

$$ u_2 = e^{\lambda t}v_2(x), \quad \lambda = 0, $$

\medskip

\begin{equation}
	v_2'' + \gamma v_2 - \rho' \ch \sqrt{-\gamma} x - \frac{3\rho^3\ch\sqrt{-\gamma}x}{4} - \frac{\rho^3\ch3\sqrt{-\gamma}x}{4} = 0,
\end{equation}
\begin{equation}
	v_2'(0) \, = 0, \quad v_2'(1) = \alpha_u\int\limits_{0}^{1} v_2(y) dy + \frac{z\sh\sqrt{-\gamma}}{\sqrt{-\gamma}}.
\end{equation}

\medskip

$$ v_2 = c\,\ch \sqrt{-\gamma} x + - \frac{z^3}{32}\ch3\sqrt{-\gamma}x + \frac{3z^3+4z'}{8\sqrt{-\gamma}}x\sh\sqrt{-\gamma}x $$
$$ c \in \mathbb{R}. $$

\end{frame}

\begin{frame}
\frametitle{ Случай дивергентной потери устойчивости }

\begin{equation}
	\rho' = \phi_0 \rho + d_0 \rho^3,
\end{equation}

\bigskip

$$ \phi_0 = 1 , $$
$$ d_0 = -\frac{5\gamma\sh3\sqrt{-\gamma}}{48\sh\sqrt{-\gamma}} - \frac{3}{4}, $$

\end{frame}

\end{document}